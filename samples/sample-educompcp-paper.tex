%% Essa classe foi adaptada do formato de conferências da ACM
\documentclass[sigconf]{educomp}

%Na versão final remover as tags review e anonymous
%\documentclass[sigconf]{educomp}

\renewcommand\refname{Referências}

%%% REMOVER ESSE PACOTE PARA A VERSAO FINAL %%%%%%%%%%%%%%%%%
\usepackage{lipsum}

%% Não altere essas informações. Elas serão enviadas aos autores após ao aceite no ato da publicação dos anais.

\copyrightyear{2024}
\acmYear{2024}
\setcopyright{sbcpt} %%% Para artigos escritos em inglês, troque para sbcen
\acmConference[EduComp'24]{Simpósio Brasileiro de Educação em Computação}{Abril 22-27, 2024}{São Paulo, São Paulo, Brasil (On-line)}
\acmBooktitle{Simpósio Brasileiro de Educação em Computação (EduComp '24), Abril 22-27, 2024, São Paulo, São Paulo, Brasil (On-line)}

%%% REMOVER A PRÓXIMA LINHA PARA ARTIGOS ESCRITOS EM INGLÊS
\usepackage[brazil]{babel}

%%%%%%%%%%
\begin{document}

%%% TITULO
\title[Título curto, caso necessário]{Um Template Para o EduComp 2024}


%Autores
\author{Ringo Star, John Lennon, Paul McCarthney, George Harrison}
\email{{ringo, john, paul, george}@apple.co.uk}
\affiliation{%
  \institution{The Fab Four, Manchester, UK}
  \country{Brasil}
}

%%
%% By default, the full list of authors will be used in the page
%% headers. Often, this list is too long, and will overlap
%% other information printed in the page headers. This command allows
%% the author to define a more concise list
%% of authors' names for this purpose.
\renewcommand{\shortauthors}{}

\newcommand{\showDOI}[1]{\unskip}


%%%%% ABSTRACT %%%%%%%%%%%%%%%%%%%%%%%%%%%%%%%
\begin{abstract}
Crie um parágrafo resumindo seu trabalho (até 1.000 caracteres, incluindo espaços). O resumo de um trabalho é importante para chamar a atenção do leitor para o que será encontrado no texto. O artigo deve ser escrito em português ou inglês. No caso da escrita em português, este \textit{template} já está configurado adequadamente. No caso da escrita em inglês, os títulos das seções e legendas de figuras e tabelas devem ser traduzidos (para utilizar o \textit{template} em LaTeX, basta seguir as instruções que estão comentadas no próprio \textit{template}). Os CCS Concepts são opcionais e as palavras-chaves são obrigatórias (de 3 a 5). A submissão para avaliação do trabalho deve ser anonimizada, ou seja, os nomes dos autores, afiliações e qualquer outro texto que permita a identificação dos autores devem ser removidos para a revisão. Por favor, fique atento(a) para o número de páginas permitido na chamada de trabalhos.
\end{abstract}

%%%% PALAVRAS-CHAVE %%%%%%%%%%%%%%
\keywords{Educação de computação, Brasil, EduComp} % germane load

%%
%% This command processes the author and affiliation and title
%% information and builds the first part of the formatted document.
\maketitle

\section{Introdução}

De acordo com \citet{bispo2020tecnologias}: 

\begin{quote}
Embora Informática na Educação (IE) e Educação de Computação (EC) partilhem muitos elementos metodológicos, epistemológicos, contextos de pesquisa e em alguns casos até objetivos de pesquisa convergentes, a EC distingue-se como área
de pesquisa específica por sua \emph{raison d’être}: a busca do entendimento profundo de fenômenos complexos e processos envolvidos no ensino e aprendizado de computação \cite{malmi2019computing}. Além disso, enquanto as TEC e a IE pressupõem o emprego de um artefato tecnológico no processo de ensino e/ou aprendizagem, a EC pode utilizar-se ou não de tais artefatos para o ensino dos processos de computação. Assim como a Ciência da Computação ainda pode ser considerada uma ciência relativamente nova e em desenvolvimento, a EC ainda encontra-se em um estado de maturação \cite{malmi2019computing}. Ainda que a maioria das teorias utilizadas para o ensino de computação seja adaptada de outras áreas de tecnologia (como a engenharia, matemática ou ciências de forma geral), ou de teorias de educação ou psicologia da educação de forma mais ampla \cite{malmi2014theoretical}, já é possível observar o advento de teorias e práticas pedagógicas próprias \cite{malmi2019computing} com um crescente suporte empírico \cite{malmi2014theoretical}
\end{quote}

\lipsum[3-10]


\section{Trabalhos Relacionados}

\lipsum[1-5]


\section{Métodos}

\lipsum[5-10]


\section{Resultados}

\lipsum[7-14]


\section{Discussão}

\lipsum[20-27]


\subsection{Limitações}

\lipsum[27-29]


\section{Conclusões}

\lipsum[29-30]


\bibliographystyle{ACM-Reference-Format}
\bibliography{references}

\end{document}
